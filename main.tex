\documentclass{article}
\usepackage[utf8]{inputenc}
\usepackage[spanish]{babel}
\usepackage{listings}
\usepackage{graphicx}
\graphicspath{ {images/} }
\usepackage{cite}

\begin{document}

\begin{titlepage}
    \begin{center}
        \vspace*{1cm}
            
        \Huge
        \textbf{Idealización proyecto final.}
            
        \vspace{0.5cm}
        \LARGE
        Informática II
            
        \vspace{1.5cm}
            
        \textbf{Juan Camilo Mazo Castro\\
        Juan Diego Carrera Quintero}
            
        \vfill
            
        \vspace{0.8cm}
            
        \Large
        Despartamento de Ingeniería Electrónica y Telecomunicaciones\\
        Universidad de Antioquia\\
        Medellín\\
        Marzo de 2021
            
    \end{center}
\end{titlepage}

\tableofcontents
\newpage
\section{Introducción}\label{intro}
En el presente informe hay dos ideas u opciones de juegos para presentar como proyecto final del curso de Informática II. Ambas ideas tienen una descripción de lo que sería el juego para dar una idea de cómo sería la mecánica.

\section{Hotel 33 (Nombre sujeto a cambios)} \label{Juego 1}
Juego survival horror estilo noventero que tiene como objetivo de escapar de un lugar de pesadillas. El juego contará con armas de fuego para evitar que los diferentes monstruos asesinen a nuestro personaje, las balas serán muy limitadas por lo que será muy conveniente usarlas de manera correcta y si el jugador se ve sin munición, la única opción que tiene es correr y esquivar. El juego podría ser con una vista desde arriba, en la que el entorno sea bastante oscuro y el personaje cuente con un farol para iluminar a su alrededor. El personaje no tendrá una gran velocidad al correr por lo que tendrá la opción de hacer saltos laterales para intentar esquivar ataques. Los entornos serán lugares bastante cerrados para generar una atmosfera más abrumadora y en la que haya que calcular muy bien los movimientos para no morir fácilmente. La vitalidad del personaje tampoco será muy alta por lo que unos cuantos ataques de monstruos y su vida se acaba. El juego será para un solo jugador ya que lo ideal sería generar terror, cosa que probablemente se reduce si es para dos jugadores. El juego debe estar muy enfocado en lo sonoro, ya que gran parte de la atmosfera aterradora de otros juegos se basa en los sonidos y como estos lo implementan. El juego será silencioso por momentos, a veces de manera repentina habrá sonidos extraños que estremezcan al jugador y tendrá música en algunos tramos que hará más tensa la situación. En ocasiones no aparecerá ningún monstruo lo que hará que el jugador no esté seguro de cuándo va ser sorprendido y esto aumenta el factor sorpresa del juego. El juego también puede ser con una vista lateral o también llamada de plataformas, en la que el jugador se mueve de manera lineal, pero en este caso puede moverse saltando hacia arriba. El juego puede desarrollarse en un hotel si la vista es desde arriba para tener un lugar mucho más cerrado y estrecho o en un lugar abierto como un bosque si la vista es lateral.\\ La historia del juego sería sobre un personaje que muere y este despierta en el infierno, pero él no sabe dónde está e intentará escapar de ese lugar pasando por un montón de obstaculos para intentar descubrir la verdad.

\newpage

\section{Elemental Dungeons} \label{Juego 2}
Elemental Dungeons o Mazmorras elementales es un juego que consiste de 4 niveles y uno extra para el final, los cuatro niveles por cada elemento (tierra, aire, agua y fuego), en cada nivel tendrás que superar cinco mazmorras donde habrán enemigos que derrotar o rompecabezas para resolver y así conseguir fragmentos que te serviran para obtener el cristal elemental el cual usarás para vencer el jefe final.

\begin{enumerate}
    \item Primer nivel (Earth Dungeon)\\
    En los calabozos de este nivel nuestro personaje deberá enfrentarse a los diferentes enemigos.
    \begin{itemize}
        \item Escarabajos de tierra.\\
        Estos son los primeros enemigos que aparecen, por lo tanto, los más débiles, no tiene ninguna habilidad más que el ataque cuerpo a cuerpo.
        \item Serpientes.\\
        Estos reptiles tienen un nivel medio de dificultad, algunas pueden envenar al jugador haciendo que pierdan vida durante unos segundos y otras pueden escabullirse por debajo de la tierra y sorprender al jugador.
        \item Golem de tierra.\\
        Es el enemigo más difícil de este nivel, aunque son lentos tienen bastante vida y su ataque es muy fuerte, pueden invocar golemitas más débiles que atacarán al jugador dificultando aún más el calabozo.
    \end{itemize}
    En cada calabozo al acabar con todos los enemigos aparecerá un elemento el cual debe obtener para avanzar al siguente.
    \begin{itemize}
        \item Llave de tierra.\\
        Esta llave aparecerá en los cuatros primeros calabozos, las cuales serán necesarias para obtener el último objeto del nivel.
        \item Fragmento Elemental de Tierra.\\
        Al llegar al quinto calabozo, conseguir las cuatro llaves de tierra y derrotar a los enemigos podrás obtener el fragmento elemental de tierra, el cual es necesario para obtener el cristal elemental.
    \end{itemize}
    Una vez obtenido el Fragmento elemental de tierra se abrirá un portal que te dirigirá al siguiente nivel.
    
    \item Segundo nivel (Air Dungeon)\\
    Al igual que en el primer nivel, el jugador deberá derrotar algunos enemigos.
    \begin{itemize}
        \item Aves.\\
        Parecen unas simples aves, pero éstas atacan en grupos, lo cual puede dificultar el nivel.
        \item Nubes cargadas.\\
        Si tocas alguna nube cargada recibirás una descarga eléctrica, la cual te quitará bastante vida, además, te aturdirá por un momento.
        \item Grifo.\\
        Este ser mitológico con sus alas crerá pequeños tornados que te aturdirán si pasas a travéz de ellos, también puede envestir al jugador haciéndole gran daño.
    \end{itemize}
    Objetos a encontrar.
    \begin{itemize}
        \item Luz solar.\\
        Estos objetos que parecen focos, pero en verdad son restos de luz solar concentrados, están repartidos en los cuatro primeros calabaozos, los cual deberás obtener para recuperar el último objeto.
        \item Fragmento Elemental de Aire.\\
        Fragmento ubicado en el último calabozo del segundo nivel el cual puede ser obtenido posicionando adecuadamente los restos de luz solar para que con su luz desbloqueen el fragmento.
    \end{itemize}
    Una vez obtenido el Fragmento elemental de aire se abrirá un portal que te dirigirá al siguiente nivel.
    
    \item Tercer nivel (Water Dungeon)\\
    Enemigos a enfrentar.
    \begin{itemize}
        \item Medusas.\\
        Si una medusa te toca te quitará vida y te aturdirá vida por unos segundos.
        \item Tiburón.\\
        Estos peces tienen una mordida fuerte, la que usarán para quitarte bastante vida.
        \item Sirena.\\
        La sirena enviará sus anguilas eléctricas para atacarte. También lazará su tridente con gran fuerza haciendo que pierdas bastante vida.
    \end{itemize}
    Objetos a encontrar.
    \begin{itemize}
        \item Burbujas.\\
        En los cuatro primeros calabozos encontrarás cuatro burbujas, necesarias para obtener el objeto final del nivel.
        \item Fragmento Elemental de Agua.\\
        Una vez derrotada a la sirena deberás usar las burbujas para que el fragmento de este nivel aparezca y así llevártelo.
    \end{itemize}
    Una vez obtenido el Fragmento elemental de agua se abrirá un portal que te dirigirá al siguiente nivel.
    
    \item Cuarto nivel (Fire Gaunlet)\\
    Enemigos.
    \begin{itemize}
        \item Demonios de fuego.\\
        Entes que no quitan demasaida vida pero al atacarte estarás en llamas por unos segundos.
        \item Fénix.\\
        Un ave que te quitará no muy poca vida y al morir hará una explosión generando dos fénix con menos vida, pero más ataque.
        \item Dragón.\\
        Este ser mitológico te atacará con su aliento de fuego y fuertes zarpazos.
    \end{itemize}
    Obejtos a encontrar.
    \begin{itemize}
        \item Obsidianas.\\
        Piedras volcánicas que deberás recolectar para obtener el objeto final del nivel.
        \item Fragmento Elemental de Fuego.\\
        Una vez derrotado el dragón usarás las obsidianas para crear el fragmento elemental de fuego y así llevártelo.
    \end{itemize}
    Una vez obtenido el Fragmento elemental de fuego se abrirá un portal que te dirigirá con el jefe final.
    
    \item Final (Aún no sé que poner de jefe final)\\
    Una vez superes los 20 calabozos o mazmorras deberás ubicar los cuatro fragmentos elementales en sus respectivos atriles, una vez en sus pocisiones aparecerá el Cristal Elemental dándote poderes elementales para así enfrentarte al jefe final.\\
    
    Los detalles de la batalla final debería pensarlos con más tiempo.
\end{enumerate}




\end{document}