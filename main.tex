\documentclass{article}
\usepackage[utf8]{inputenc}
\usepackage[spanish]{babel}
\usepackage{listings}
\usepackage{graphicx}
\graphicspath{ {images/} }
\usepackage{cite}

\begin{document}

\begin{titlepage}
    \begin{center}
        \vspace*{1cm}
            
        \Huge
        \textbf{Idealización proyecto final.}
            
        \vspace{0.5cm}
        \LARGE
        Informática II
            
        \vspace{1.5cm}
            
        \textbf{Juan Camilo Mazo Castro}
            
        \vfill
            
        \vspace{0.8cm}
            
        \Large
        Despartamento de Ingeniería Electrónica y Telecomunicaciones\\
        Universidad de Antioquia\\
        Medellín\\
        Marzo de 2021
            
    \end{center}
\end{titlepage}

\tableofcontents
\newpage
\section{Sección introductoria}\label{intro}
El presente informe se realiza con el objetivo de dar una serie de ideas para la creación del videojuego que hace parte del proyecto final para el curso de Informática II. Se hará una descripción de varias de las posibles caracteristicas que tendra el videojuego.

\section{Sección de contenido} \label{contenido}
Juego survival horror estilo noventero que tiene como objetivo de escapar de un lugar de pesadillas. El juego contará con armas de fuego para evitar que los diferentes monstruos asesinen a nuestro personaje, las balas serán muy limitadas por lo que será muy conveniente usarlas de manera correcta y si el jugador se ve sin munición, la única opción que tiene es correr y esquivar. El juego podría ser con una vista desde arriba, en la que el entorno sea bastante oscuro y el personaje cuente con un farol para iluminar a su alrededor. El personaje no tendrá una gran velocidad al correr por lo que tendrá la opción de hacer saltos laterales para intentar esquivar ataques. Los entornos serán lugares bastante cerrados para generar una atmosfera más abrumadora y en la que haya que calcular muy bien los movimientos para no morir fácilmente. La vitalidad del personaje tampoco será muy alta por lo que unos cuantos ataques de monstruos y su vida se acaba. El juego será para un solo jugador ya que lo ideal sería generar terror, cosa que probablemente se reduce si es para dos jugadores. El juego debe estar muy enfocado en lo sonoro, ya que gran parte de la atmosfera aterradora de otros juegos se basa en los sonidos y como estos lo implementan. El juego será silencioso por momentos, a veces de manera repentina habrá sonidos extraños que estremezcan al jugador y tendrá música en algunos tramos que hará más tensa la situación. En ocasiones no aparecerá ningún monstruo lo que hará que el jugador no esté seguro de cuándo va ser sorprendido y esto aumenta el factor sorpresa del juego. El juego también puede ser con una vista lateral o también llamada de plataformas, en la que el jugador se mueve de manera lineal, pero en este caso puede moverse saltando hacia arriba. El juego puede desarrollarse en un hotel si la vista es desde arriba para tener un lugar mucho más cerrado y estrecho o en un lugar abierto como un bosque si la vista es lateral.\\ La historia del juego sería sobre un personaje que muere y este despierta en el infierno, pero él no sabe dónde está e intentará escapar de ese lugar pasando por un montón de obstaculos para intentar descubrir la verdad.




\end{document}
